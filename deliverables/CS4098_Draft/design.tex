\subsection{Data Flow}

\subsubsection{Input and Processing}
Multiple formats can be processed using the command line tool that has been developed during this project. Initially only TCPdump was considered, the tool took input in the form of binary PCAP files, this was then extended to allow syslog formats to be used as input. The output produced is in an identical format regardless of the input type. Data flow for the two input types is similar and detailed in figure \ref{}.\newline

The first stage of processing is to extract associations between devices and access points from the raw input. These associations will later be compared with each other to determine encounters between devices, these comparisons are expensive and so it is important that as much information as possible is removed before they are made. The intermediate format used to store association information is  a comma separated value file with fields of source id, destination id, start time, end time, and AP flag. The AP flag is set to 1 only if the destination address of the association can be identified as an access point. Methods used to identify access points are discussed later in this report.

TCPdump output PCAP files are very strictly structured. This allows for them to be processed without any additional information from the user. However the initial parsing is expensive with respect to time since so much information is contained within them. In comparison to this, syslog files are relatively quick to parse for associations, but the user must include a configuration file to specify the format used. Details such as defining features of association end points and the format of device identification values need to be given to the tool before it can extract associations. The expected format of a configuration file for use with syslog is detailed in figure \ref{}.\newline

Tying together the components of this data flow is a BASH script. Temporary files are used for storing associations in the intermediate stages, these files are then read by the MatLab script which compares associations to find encounters. A final output file is then given as a comma separated values file. 
% Optional levels of abstraction??

% explain fields chosen
%   - how is encounter data used in external studies?
% Data flow diagrams
% BASH scripting

\subsubsection{Summary Output}
The final output specifies the endpoints of encounters, the average time of encounters between the given end points, and the number of encounters found between the given end points. Only one entry in the CSV file is present for each unordered pair of end points.\newline

The data sets which I am using in this project have in the past frequently been used for research into mobility and encounter patterns between users \cite{Scellato2011} \cite{Xiao2014} \cite{Hsu2010} \cite{Musolesi2009} \cite{Kosta2014} \cite{Kumar2009} \cite{Wei2013}. To maximise the usefulness of this project it is intended that the output from this  summarisation tool will be able to help in identifying whether a dataset is appropriate for use in mobility research, and to provide a standard format between multiple input formats which can be used for onward processing.

\subsection{Associations} 
%(what? why?)
\subsubsection{Identifying Access Points}

\subsubsection{Initiation} 
% (what? why?)
% options
% decision

\subsubsection{Timeout} 
% (what? why?)
% research
% decision
% alternatives? why not?

\subsection{Encounters} 
% (what? why?)
% Encounters from associations


\subsection{UI} % or lack of...
