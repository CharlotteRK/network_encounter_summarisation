\subsubsection{Existing Formats for Aggregation of Network Traces in Mobility Research}

A lack of published information on the intermediate formats used while analysing network traces for mobility research has been found during this survey. This is likely due to the encounter data not being the final outcome of the research taking place and therefore not being considered important enough to write up.

Through varied searches of DTN, Mobility, and SDN research I have found only two examples of well documented formats for storing data on device encounters. The first of these, The ONE Simulator \cite{Keranen2009}, uses several reporting options to store device encounter data. The other documented format (from \cite{Thakur2012}) that was found was an association matrix. These two sources and and analysis of the information found in them is set out in the following section of this report.

\subsubsection{The ONE Simulator \cite{Keranen2009}}
The ONE is a simulator which generates data intended to mimic a network of mobile nodes. It then reports this data using various reporting modules, three of these modules focus on data regarding encounters between devices.

The first and most simple of these reports contains information about the dispersion of the total number of encounters experienced by the nodes in the network. It consists of two fields, one containing the number of encounters, and the other containing the count of nodes that have experienced that number of encounters. This contains no information relating to the unique nodes between which the encounters occur, or any temporal information such as duration of the encounters. 

The second format provides information on the uniqueness of the encounters recorded, but loses detail about the total number of encounters in the dataset. This format also contains two data fields, one containing the values from 0 to 1000; representing promilles. The second field contains the number of unique pairs encountering with frequency withing the corresponding promille. This has a benefit of being almost static in size as the number of nodes int he system increases. 

The final report format from the ONE simulation which has been looked at is a combination of the two previously discussed reports. It has three fields; the first contains an identifier for each node, the second contains the total number of encounters that the node has had, and the third contains the number of unique nodes with which it has had an encounter. This still does not uniquely identify both devices in an encounter, nor does it provide and detail about the duration of the encounters.

\subsubsection{Association Matrices}
\citeauthor{Thakur2012} use an association matrix to record the percentage of time each node spends in an encounter with each other node. A matrix is created for each node, each column in the matrix corresponds to the other endpoint of the encounter, and each row corresponds to a time interval. The entry in each cell represents the percentage of the time interval spent in an encounter with the columns node. This format contains the most information out of all those discussed here, however also takes more space. The space taken will increase at with the square of the number of nodes.

\begin{table}
\small
\begin{tabular}{|p{5.5cm}|p{1.5cm}|p{1.5cm}|p{1.5cm}|}
\cline{3-4}
\multicolumn{1}{}{} & \multicolumn{3}{|p{4.5cm}|}{Level of Detail (Complete, Most, Some, or None)} \\ \hline
 Format&Endpoints&Duration&Frequency\\ \hline
 TotalEncountersReport - The ONE \cite{Keranen2009}&Some&None&Complete \\ \hline
 UniqueEncountersReport - The ONE \cite{Keranen2009}&Some&None&Most \\ \hline
 EncountersVsUniqueEncounters - The ONE \cite{Keranen2009}&Most&None&Complete \\ \hline
 Association Matrix \cite{Thakur2012}&Complete&Some&Some \\ \hline
 
\end{tabular}
\caption{Table of existing formats for storing data about device encounters and the level of detail they contain regarding the unique endpoints and length/frequency of the encounters.}
\label{formatTable}
\end{table}


\subsubsection{Summary of Findings}
A very brief summary of the detail contained within each of the formats discussed above is given in Table \ref{formatTable}. Despite association matrices storing the most useful information, the polynomial increase in size with the number of mobile nodes makes using them potentially ineffective in the context of this project. The aim is to summarize a large amount of data into a smaller, easier to process format. In many cases association matrices would decrease the size of the data, but by a dramatically lesser amount than the other formats discussed here. Additionally it would be possible for the association matrix format to increase the quantity of data; for instance if $N$ nodes had $< N$ encounters each then the association matrix for each node would include at least one redundant column. The final reporting format discussed under the ONE simulation - EncountersVsUniqueEncounters - avoids this polynomial growth, with its size increasing only linearly with the number of mobile nodes in the network. It provides less complete information regarding the unique par of nodes between which the encounter occurred, there would however be no way to preserve this information while avoiding at least $N^{2}$ growth in size.

It seems that the most complete format of those discussed in which to store encounter data while also guaranteeing a reduction in the quantity of data stored would be the EncountersVsUniqueEncounters report format. This format could also be easily modified to add additional fields such as statistics regarding encounter duration. Any additional fields would need to be carefully considered and justified in order to keep the data quantity reduction as high as possible.