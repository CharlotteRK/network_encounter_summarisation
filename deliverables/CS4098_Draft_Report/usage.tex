\subsection{Research}

When completing this research the focus has been on one particular dataset from the CRAWDAD archive, dartmouth/campus \cite{dartmouthcampus2009}. This dataset was chosen because it is one of the most popular datasets in the archive, having been cited by 374 papers at the time of writing \cite{aboutCrawdad2014}. The most frequently cited dataset however is cambridge/haggle, the reasoning for deciding not to focus on this instead is that the Cambridge dataset is comparatively small in size and would therefore benefit much less from the summarisation which this project hopes to provide.
\newline
The papers that have been selected for use in this research were chosen because they all cite the dartmouth/campus dataset. A Google Scholar \cite{googleScholar} online search was used to retrieve the most "relevant" papers which used the chosen dataset, from these results the ones which have been most often cited in other work were selected. This selection process found papers which are relevant in the research community. As there are different versions of the dataset the search had to be repeated three times, once using the 2009 dataset, once with the 2007 dataset, and once with the 2005 dataset. For each search the five most cited results have been used. Table \ref{usageTable} shows a summary of the type of information each paper needed to use from the dartmouth/campus dataset. Papers in which the dataset was referenced but ultimately has not been used have been excluded.

\small
\begin{longtable}{|p{3cm}|p{1.9cm}|p{1.7cm}|p{1.5cm}|p{1.9cm}|}
\cline{3-5}
\multicolumn{2}{}{} & \multicolumn{3}{|p{4.5cm}|}{Properties Needed} \\ \hline
Paper                                                & Topic                              & Device/AP Identification   & Time of Transmission  & Transmission Quality/Rate  \\ \hline
%308
Nextplace: a spatio-temporal prediction framework for pervasive systems, \citeauthor{Scellato2011}, \citeyear{Scellato2011}   & Mobility                           & x                          & x                     &                            \\ \hline
%191
Community-Aware Opportunistic Routing in Mobile Social Networks, \citeauthor{Xiao2014}, \citeyear{Xiao2014}           & Mobility                           & x                          & x                     &                            \\ \hline
%155
On nodal encounter patterns in wireless LAN traces, \citeauthor{Hsu2010}, \citeyear{Hsu2010}             & Mobility                           & x                          & x                     &                            \\ \hline
%140
Mobility models for systems evaluation, \citeauthor{Musolesi2009}, \citeyear{Musolesi2009}   & DTN                                & x                          & x                     &                            \\ \hline
%60
Large-Scale Synthetic Social Mobile Networks with SWIM, \citeauthor{Kosta2014}, \citeyear{Kosta2014}         & Mobility                           & x                          & x                     &                            \\ \hline
%27
WAVEFORM DESIGN AND NETWORK SELECTION IN WIDEBAND SMALL CELL NETWORKS, \citeauthor{Yang2014}, \citeyear{Yang2014}           & Mobility                           & x                          &                       & x                          \\ \hline
%18
MAGA: A Mobility-Aware Computation Offloading Decision for Distributed Mobile Cloud Computing, \citeauthor{Shi2018}, \citeyear{Shi2018}             & Mobility                           & x                          & x                     &                            \\ \hline
%18
Flow-Based Management For Energy Efficient Campus Networks, \citeauthor{Amokrane2015}, \citeyear{Amokrane2015}   & SDN                                & x                          &                       & x                          \\ \hline
%16
Human behavior and challenges of anonymizing WLAN traces, \citeauthor{Kumar2009}, \citeyear{Kumar2009}         & Anonymizing WLAN Traces            & x                          & x                     &                            \\ \hline
%15
Automatic profiling of network event sequences: algorithm and applications, \citeauthor{Meng2008}, \citeyear{Meng2008}           &Profiling of Network Event Sequences& x                          & x                     &                            \\ \hline
%13
Confidentiality of event data in policy-based monitoring, \citeauthor{Montanari2012}, \citeyear{Montanari2012} & Policy-Based Monitoring            & x                          &                       &                            \\ \hline
%12
Distribution of inter-contact time: An analysis-based on social relationships, \citeauthor{Wei2013}, \citeyear{Wei2013}             & Distribution of Inter-Contact Time & x                          & x                     &                            \\ \hline
%12
Coverage and Rate Analysis for Facilitating Machine-to-Machine Communication in LTE-A Networks Using Device-to-Device Communication, \citeauthor{Swain2017}, \citeyear{Swain2017}         & Machine-to-Machine Communication   & x                          & x                     &                            \\ \hline
%10
Balancing reliability and utilization in dynamic spectrum access, \citeauthor{Cao2012}, \citeyear{Cao2012}             & Dynamic Spectrum Access            & x                          & x                     &                            \\ \hline
%9
An Online Algorithm for Task Offloading in Heterogeneous Mobile Clouds, \citeauthor{Zhou2018}, \citeyear{Zhou2018}           & Offloading                         & x                          & x                     &                            \\ \hline
%6
State-of-the-Art Routing Protocols for Delay Tolerant Networks, \citeauthor{Feng2012}, \citeyear{Feng2012}          & DTN                                 & x                          &                       & x                          \\ \hline

\captionsetup{width=\textwidth}
\caption{Table of the properties of CRAWDAD dartmouth/campus data used in various research projects in which it was cited. Papers are ordered by the number of other papers they have been cited by, with the most cited at the top.}
\label{usageTable}
\end{longtable}

\subsection{Summary of Results}
The usage of the Dartmouth University CRAWDAD dataset is primarily regarding network mobility and social interaction/encounters. As such, the most often needed information seems to be identifiers for both mobile devices and access points, and the times of connections.
I found that the majority of the papers I looked at used the movement \cite{dartmouthcampus2009movement} or syslog \cite{dartmouthcampus2009syslog} tracesets as these are most tailored towards mobility research. 

There are also some less frequent topics of research such as software defined networking and delay tolerant networking using the dartmouth/campus dataset. These uses seem to require a wider variety of information from the data, however these instances are much less frequent than those mentioned above. these less common cases are the only ones which mention bandwidth and quality of connection.

