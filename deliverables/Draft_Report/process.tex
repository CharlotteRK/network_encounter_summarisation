%Time plan
During this project several separate development cycles were completed, each with a different end goal. Each cycle began with a research and planning phase which then lead to a implementation and testing phase. Testing was done continuously during the implementation of features, with additional performance testing done at the end of each implementation phase once all new features had been added.
Each week throughout the project a selection of tasks was made. At time flexibility was needed as some tasks were expected to take more than one week to complete. In such cases tasks were selected multiple weeks in a row.\\\\
As a final stage in each development cycle the product is evaluated against the requirements of this project. This gives a starting point for the next cycle, requirements which are fully met do not need to be implemented again in the next cycle, but still need to be given consideration to ensure future changes do not compromise them. \\\\
Throughout this project the school's Mercurial system has been used for version control. The Mercurial repository was use not only for the source code of the the summarisation tool but also to enable me to keep track of my own progress through weekly logging of the tasks which were completed and any issues that were encountered. By keeping track of past performance future weeks workloads were able to be more efficiently and realistically planned.S
\subsection{Development Cycle One: MVP}
The first development cycle was intended to be completed before the start of the second semester and to produce a fully working, well tested MVP which met all the high priority objectives set out in section 4 of this report.  
    %Research
    \subsubsection{Research and Planning}
    The first few weeks of this cycle were spent researching other work in the area, and completing the context survey as detailed in section three of this report. Then, using the findings from the research, the intended output format was specified. An outline for the stages needed during the processing of the data was decided and appropriate languages and libraries for each stage were decided on.
    %Development cycle 1 (MVP)
    \subsubsection{Implementation and Testing}
    Features that were implemented during this cycle were:
    \begin{itemize}
        \item The ability to take tcpdump output as input
        \item Extraction of encounters from input data into a simple CSV format 
        \item A simple command line interface allowing and input directory to be specified
        \item A BASH script tying together the three stages of the MVP's data flow
    \end{itemize}
    Once these features were implemented through tests were done using the CRAWDAD dartmouth data. The output was considered to be reasonable by inspection of the number of access points detected by the summarisation tool. In addition the length of associations and encounters were checked to make sure they were realistic in the context of the building in which the data was collected. Complete accuracy was difficult to ensure since there is not existing encounter data to compare this output to.\\\\
    After completing some testing of the time efficiency of the data processing showed that this initial implementation was unreasonably slow under certain conditions. Specifically when high numbers of access points were used, the matching of associations into encounters took a long time to run. Once five access points were detected the process would take several hours to run, while the same length of input with only two distinct access points would complete processing in around twenty minutes. Speeding this up is something considered in the next cycle.
    \subsubsection{Evaluation Against Requirements}
    The product developed by the end of this cycle satisfies the first two of the high priority requirements identified in section 4. The quantity of data is significantly less in the output of the product than the input. Research was done which identified encounters between devices as the most useful information in the data, so this information was kept in the output. The third high priority requirement was only partially met at this stage; the CSV format used for output has several benefits (such as its simplicity and flexibility) however no evidence has been found to support its use in the context to encounter traces in existing mobility research.   
\subsection{Development Cycle Two: Extending Capabilities}
In this second cycle the aim was to find a more justifiable output format, to improve the time efficiency of the existing code, and to add features which extend the functionality of the product. The medium priority requirements in section 4 should be considered while adding functionality.
    %Research
    \subsubsection{Research and Planning}
    The research in this cycle was mainly technical topics. A better understanding of the MatLab language was needed in order to write more efficient code, and so further research into the strengths and weaknesses of the language was undertaken. The ONE simulator was also further researched and how the output of this project could be altered to be compatible with it.
    %Development cycle 2 (Extending)
    \subsubsection{Implementation and Testing}
    Updated and improved functionality:
    \begin{itemize}
        \item The speed dependency of the processing on the number of access points was reduced from a quadratic relationship to a linear one
        %\item The output of the product was altered to be compatible with the ONE simulator [TODO!]
    \end{itemize}
    These additional features were added:
    \begin{itemize}
        \item Capability to parse syslog files in addition to tcpdump
        %\item Multiple summaries can be merged to create a single output of similar format [TODO!]
        %\item A summary may now be updated by a single data entry [TODO, possibly won't get time]
    \end{itemize}
    By using appropriate data structure in MatLab the languages efficient referencing and searching was able to be used. This greatly reduced processing time when the intermediate associations data needs to be matched against known access points. It was this efficiency of MatLab that made it a justifiable choice for implementing the transformation from associations to encounters, however an inefficient method of looping through associations had mistakenly been used in the first implementation.
    \subsubsection{Evaluation Against Requirements}
    During this development cycle one additional medium priority requirement has been met (ability to summarise multiple known input formats). In addition to this the performance of the summarisation tool has been improved, and progress has been made toward using a more suitable output format which will be compatible with the ONE simulator.
    
%Ongoing change
    %

%Testing
    %

