In order of priority the requirements of this project have been listed below. These are the same requirements as set out in the DOER document submitted at the beginning of this project, and have been described in greater detail here. Each of these requirements can be implemented separately from one another, and with the completion of all high priority requirements a working product will be completed. The high priority requirements combine to describe a minimum viable product to meet the overreaching aims of this project. \\
Implementation of the medium and low priority requirements will complete a more versatile system which could be applied to datasets with a wider variety of sizes and formats.\\
All of the requirements set out here are functional requirements, in that they describe features to be implemented in order to produce a high quality summarisation tool.

\subsection{High Priority}
\begin{itemize}
    \item Reduce the quantity of data from the original dataset while maintaining any information identified as useful during research into how CRAWDAD datasets are used.
    \begin{itemize}
        \item During the context survey it was found that the most used information from these datasets was the encounters between different devices. Mobile node encounters (including the nodes involved and duration) will need to be included in the summaries for this requirement to be met. 
        \item It is important that the total quantity of data is reduced in the summary compared to the original dataset. For this requirement to be met it should be shown that the output is guaranteed to contain less data than the input.
    \end{itemize}
    \item Produce summaries of the initial datasets that can be processed more efficiently than the original data.
    \begin{itemize}
        \item In order to meet this objective it is necessary for the summary format to be as simple as possible. Superfluous information and complex file types will need to be avoided. 
    \end{itemize}
    \item Use a commonly found format to output my summaries and justify why this format is appropriate in context.
    \begin{itemize}
        \item In section 7 (Design) a complete output format will be specified and justification given based on the research in section 3.2.
        \item Specification of the output summary format should include the file type that will be used, the fields that will be included, and the variable types that will be used. All of these decisions will need to be supported by relevant and reliable research.
    \end{itemize}
\end{itemize}

\subsection{Medium Priority}
\begin{itemize}
    \item Allow multiple summaries to be merged (this may allow extension into distributed processing).
    \begin{itemize}
        \item This should allow two summaries which have been output by the minimum viable product to be given as arguments at run time and combined into a single summary.
        \item The output of this will use the same format as the input summaries, as if created by running the basic program on the combined network traces. 
    \end{itemize}
    \item Summarise at least two different formats of input data to create a standard output summary.
    \begin{itemize}
        \item The system will support summarising more than one input format, but the input format should be specified at run time.
        \item Whichever format the input takes the same output format should be produced. 
    \end{itemize}
    \item Allow a summary to be updated by the addition of a single data entry (this may allow extension into real-time processing).
    \begin{itemize}
        \item This requirement will be met if a summary can have added to it a single item of data (the exact definition of which will depend on the input format of the data, for instance a single data packet transfer in a tcpdump trace). With the resulting summary being identical to the summary which would have resulted from an initial input which included the new data entry.
    \end{itemize}
\end{itemize}

\subsection{Low Priority}
\begin{itemize}
    \item Process datasets with an unknown input format.
    \begin{itemize}
        \item The format of the input data should not need to be specified at run time.
        \item Multiple input formats should be accepted and and the system should be able to differentiate between them in order to process each correctly.
    \end{itemize}
    \item Identify and report if a specific summary is likely to be unrepresentative of the input dataset due to aspects such as missing data or bias.
    \begin{itemize}
        \item Depending on the information included in the summary, how representative it is may be effected for various reasons. Due to this it is important that any information used to determine whether a summary is unrepresentative is justified.
    \end{itemize}
\end{itemize}