%Aim:
This project aims to create a tool which is able to take large data sets from the collection of information over wireless networks and reduce the information content for easier onward processing. Wireless network traces are used in a wide variety of research areas and it would be impractical to consider each and every use case. The CRAWDAD archive contains some frequently used datasets of large scale which will be the focus of this project, however the tool may be used with other data as long as it is in tcpdump or syslog format. The past usage of the CRAWDAD datasets shows popularity mainly in network mobility research. Taking this into consideration the tool being developed will preserve information about encounters between mobile nodes in the network.

%Motivation:
Encounter traces are used in several different areas of research. A tool which allows network data to be summarised into a format containing significantly less superfluous information could be beneficial to researchers since it will minimise the time needing to be spent extracting relevant information from the initial datasets. By using an output format which is already popularly used 

%Overview of research:
As previously mentioned, the data which is being used as a initial focus for this project is primarily used for mobility research. Encounter information has been identified as the most beneficial information to retain from the initial input data. Research into existing intermediate formats used in mobility research proved frustrating; most research papers which were analysed gave vague (if any) descriptions of the data formats used. A common thing across several papers in multiple research areas was the use of the Opportunistic Network Environment (ONE) simulator. It would therefore be convenient for the output of the newly developed tool to be compatible with the ONE simulator. This would allow onward processing using the simulator to be undertaken by researchers. 

%Brief talk about what I'll do:
During this project the CRAWDAD dartmouth/campus dataset \cite{dartmouthcampus2009} has been the initial focus of development. Early decisions were made based on what would be most beneficial for use with the dartmouth/campus data. This was because the dataset was identified early on as one of the most frequently used datasets regarding wireless network traces in the CRAWDAD archive. Despite this early focus on the dartmouth/campus data the data summarisation tool developed throughout this project is intended for use with any data of appropriate input format. Input formats which the tool accepts are tcpdump and syslog, although syslog input requires additional configuration information to be given. These are two extremely common formats for collection of network data and were chosen based on their availability and popularity. In addition to these two formats being compatible, the data flow has been separated into multiple processing stages to minimise the work needed in order to allow processing of new data formats.
The output of the tool which has been developed is intended to facilitate easy onward processing of the summarised data. In order to meet this goal the output format is such that it can be used with the Opportunistic Network Environment simulator. The ONE simulator is commonly referenced in research papers relating to wireless networking and can take data in a strict format as input to define parameters of its simulation. To maximise the usefulness of this projects output is formatted so that it may be used as input for one of these simulations.
The final tool produced during this project meets all of the high priority requirements as specified in section 4 of this document. The majority of the medium priority requirements are also met. A full evaluation of the satisfaction of these requirements, and a critical assessment of the final tool that has been developed, is given at the end of this report.
